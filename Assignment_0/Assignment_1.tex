\documentclass{article}
\usepackage[utf8]{inputenc}
\usepackage{amsmath}
\usepackage{amssymb}
\usepackage{todonotes}
\usepackage{hyperref}
\usepackage{enumitem}

\title{Assignment 0}
\date{August 2020}

\begin{document}

\maketitle
The objective of this lab assignment is to familiarize you with the \texttt{numpy} and \texttt{matplotlib} libraries in \texttt{python}.

\todo{We need to give clear instructions on how to submit and what/format to submit That was a constant complaint last year.}

% Broad topics to cover - 
% \begin{enumerate}
%     \item Vectorization in numpy (Can be done by comparing runtimes of for loop and vectorized code)
%     \item Converting equations into numpy functions (can be combined with above point
%     \item Plotting with different scales?
%     \item Some probability or lin-alg thing?
% \end{enumerate}

\section{Problem on Probability}
\subsection{Theoretical Problems}
\begin{enumerate}[leftmargin=*, label=(\alph*)]
\item Let $X \sim F_X$ be a random variable with given CDF $F_X$. Let there be another random variable $Y \sim Unif[0,1]$. Express $X$ in terms of $Y$ and prove your result.
\item Let $X$ be an \href{https://en.wikipedia.org/wiki/Exponential_distribution}{exponential distribution} with parameter $\lambda$. Using the above result, obtain $X$ in terms of $Y$.
\end{enumerate}
\subsection{Programming}
Let $X$ be from the exponential distribution with $\lambda = 1.0$. Complete the function \texttt{modify\_uniform} in the file \texttt{probability.py} to generate samples of this variable using only the \texttt{numpy.random.uniform} function. \\
\todo{Alternatively, we can have this $X$ to be the output of a softmax and ask to generate samples for that.}  
% \subsection{Observation and plotting}
% Change $\lambda$ etc and generate scatter plots?

\section{Vectorization}
In this problem you will implement a function to compute the pair-wise $L_2$ similarity between each pair of points in a set. \\
Let $X \in \mathbb{R}^{n \times d}$ where $n$ is the number of points in the the set and $d$ is the number of dimensions of the basis if the points, the $L_2$ similarity between two points $x$ and $y$ is defined as 
\todo{Even if problem 2 is only about kernel\_vec and its efficiency, wont the concept of kernel itself be a bit intimidating as a first assignment? Especially given it is all remote}

\begin{equation*}
    d(x,y) = \sum_{i=1}^d (x_i - y_i)^2
\end{equation*}

% \subsection{Theoretical Problems}

\begin{enumerate}[leftmargin=*, label=(\alph*)]
\item Obtain a vector expression for $d(x,y)$ when $x , y \in \mathbb{R}^d$
% \item Given the a sets of $n$ points, we are interested in obtaining a matrix $K \in \mathbb{R}^{n_1 \times n_2}$ where $K_{ij} = d(X_i,X_j)$. \\\\
% Express $K$ using $X$ and the following matrix operations -  (addition, subtraction, multiplication, transpose)
% \end{enumerate}
% \subsection{Programming Problems}
% \begin{enumerate}[leftmargin=*, label=(\alph*)]
\item Complete the function \texttt{pairwise\_similarity\_looped} in the file \texttt{similarity.py} to obtain this matrix $K$ using for loops. \\

The above $\mathcal{O}(n^2 d)$ solution doesn't scale well. However, \texttt{numpy} provides a powerful mechanism called vectorization which can speed up this process drastically. \\
\todo{Soumya: We might ask them to time it for different n and d}

\item Complete the function \texttt{pairwise\_similarity\_vec} in the file \texttt{similarity.py} to obtain $K$ in a vectorized manner. Refer to the comments in the function for an approach to this problem.

% \item For each point in $X$, compute the index of the closest point in $Y$ and the distance from that point. Do the same for $Y$. Complete the function
% \texttt{closest\_point}.
\end{enumerate}
% 4. Zero out entries which are lower than a threshold (boolean and masking)?


\section{Probability and simulation}
% Here we can put a problem where we discuss a monte carlo simulation, having some sort of  a game and getting students to compute the expected reward theoretically and verify it with a numpy simulation.

You are given a special coin for which probability of getting a Head is 0.25 and probability of getting a Tail is 0.75. You are told to keep flipping the coin till you get two consecutive heads. What is the expected number of flips that you have to make? 
\begin{enumerate}[leftmargin=*, label=(\alph*)]
    \item Compute the expected value  analytically.
    \item Write a \texttt{numpy} program in the file \texttt{simulation.py} to simulate this experiment for $n=10,100,1000,10000$ steps and plot a graph of the observed expected value vs number of steps. Report your observations.
\end{enumerate}
%  for a large number of times and report the expected value you observe. Also 

\section*{Submission Instructions}
% We have provided a skeleton code in the folder \texttt{}. Complete the code in the files a.py,b.py,c.py,d.py, and add solutions to theoretical questions as a SINGLE pdf document named rollnumber_sol.pdf. Gzip the folder using - \texttt{tar -czvf <rollnumber>\_asmt0.tar.gz <rollnumber>\_asmt0} and submit.   
\end{document}
